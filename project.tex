\documentclass[11pt]{article}

    \usepackage[breakable]{tcolorbox}
    \usepackage{parskip} % Stop auto-indenting (to mimic markdown behaviour)
    

    % Basic figure setup, for now with no caption control since it's done
    % automatically by Pandoc (which extracts ![](path) syntax from Markdown).
    \usepackage{graphicx}
    % Maintain compatibility with old templates. Remove in nbconvert 6.0
    \let\Oldincludegraphics\includegraphics
    % Ensure that by default, figures have no caption (until we provide a
    % proper Figure object with a Caption API and a way to capture that
    % in the conversion process - todo).
    \usepackage{caption}
    \DeclareCaptionFormat{nocaption}{}
    \captionsetup{format=nocaption,aboveskip=0pt,belowskip=0pt}

    \usepackage{float}
    \floatplacement{figure}{H} % forces figures to be placed at the correct location
    \usepackage{xcolor} % Allow colors to be defined
    \usepackage{enumerate} % Needed for markdown enumerations to work
    \usepackage{geometry} % Used to adjust the document margins
    \usepackage{amsmath} % Equations
    \usepackage{amssymb} % Equations
    \usepackage{textcomp} % defines textquotesingle
    % Hack from http://tex.stackexchange.com/a/47451/13684:
    \AtBeginDocument{%
        \def\PYZsq{\textquotesingle}% Upright quotes in Pygmentized code
    }
    \usepackage{upquote} % Upright quotes for verbatim code
    \usepackage{eurosym} % defines \euro

    \usepackage{iftex}
    \ifPDFTeX
        \usepackage[T1]{fontenc}
        \IfFileExists{alphabeta.sty}{
              \usepackage{alphabeta}
          }{
              \usepackage[mathletters]{ucs}
              \usepackage[utf8x]{inputenc}
          }
    \else
        \usepackage{fontspec}
        \usepackage{unicode-math}
    \fi

    \usepackage{fancyvrb} % verbatim replacement that allows latex
    \usepackage{grffile} % extends the file name processing of package graphics
                         % to support a larger range
    \makeatletter % fix for old versions of grffile with XeLaTeX
    \@ifpackagelater{grffile}{2019/11/01}
    {
      % Do nothing on new versions
    }
    {
      \def\Gread@@xetex#1{%
        \IfFileExists{"\Gin@base".bb}%
        {\Gread@eps{\Gin@base.bb}}%
        {\Gread@@xetex@aux#1}%
      }
    }
    \makeatother
    \usepackage[Export]{adjustbox} % Used to constrain images to a maximum size
    \adjustboxset{max size={0.9\linewidth}{0.9\paperheight}}

    % The hyperref package gives us a pdf with properly built
    % internal navigation ('pdf bookmarks' for the table of contents,
    % internal cross-reference links, web links for URLs, etc.)
    \usepackage{hyperref}
    % The default LaTeX title has an obnoxious amount of whitespace. By default,
    % titling removes some of it. It also provides customization options.
    \usepackage{titling}
    \usepackage{longtable} % longtable support required by pandoc >1.10
    \usepackage{booktabs}  % table support for pandoc > 1.12.2
    \usepackage{array}     % table support for pandoc >= 2.11.3
    \usepackage{calc}      % table minipage width calculation for pandoc >= 2.11.1
    \usepackage[inline]{enumitem} % IRkernel/repr support (it uses the enumerate* environment)
    \usepackage[normalem]{ulem} % ulem is needed to support strikethroughs (\sout)
                                % normalem makes italics be italics, not underlines
    \usepackage{mathrsfs}
    

    
    % Colors for the hyperref package
    \definecolor{urlcolor}{rgb}{0,.145,.698}
    \definecolor{linkcolor}{rgb}{.71,0.21,0.01}
    \definecolor{citecolor}{rgb}{.12,.54,.11}

    % ANSI colors
    \definecolor{ansi-black}{HTML}{3E424D}
    \definecolor{ansi-black-intense}{HTML}{282C36}
    \definecolor{ansi-red}{HTML}{E75C58}
    \definecolor{ansi-red-intense}{HTML}{B22B31}
    \definecolor{ansi-green}{HTML}{00A250}
    \definecolor{ansi-green-intense}{HTML}{007427}
    \definecolor{ansi-yellow}{HTML}{DDB62B}
    \definecolor{ansi-yellow-intense}{HTML}{B27D12}
    \definecolor{ansi-blue}{HTML}{208FFB}
    \definecolor{ansi-blue-intense}{HTML}{0065CA}
    \definecolor{ansi-magenta}{HTML}{D160C4}
    \definecolor{ansi-magenta-intense}{HTML}{A03196}
    \definecolor{ansi-cyan}{HTML}{60C6C8}
    \definecolor{ansi-cyan-intense}{HTML}{258F8F}
    \definecolor{ansi-white}{HTML}{C5C1B4}
    \definecolor{ansi-white-intense}{HTML}{A1A6B2}
    \definecolor{ansi-default-inverse-fg}{HTML}{FFFFFF}
    \definecolor{ansi-default-inverse-bg}{HTML}{000000}

    % common color for the border for error outputs.
    \definecolor{outerrorbackground}{HTML}{FFDFDF}

    % commands and environments needed by pandoc snippets
    % extracted from the output of `pandoc -s`
    \providecommand{\tightlist}{%
      \setlength{\itemsep}{0pt}\setlength{\parskip}{0pt}}
    \DefineVerbatimEnvironment{Highlighting}{Verbatim}{commandchars=\\\{\}}
    % Add ',fontsize=\small' for more characters per line
    \newenvironment{Shaded}{}{}
    \newcommand{\KeywordTok}[1]{\textcolor[rgb]{0.00,0.44,0.13}{\textbf{{#1}}}}
    \newcommand{\DataTypeTok}[1]{\textcolor[rgb]{0.56,0.13,0.00}{{#1}}}
    \newcommand{\DecValTok}[1]{\textcolor[rgb]{0.25,0.63,0.44}{{#1}}}
    \newcommand{\BaseNTok}[1]{\textcolor[rgb]{0.25,0.63,0.44}{{#1}}}
    \newcommand{\FloatTok}[1]{\textcolor[rgb]{0.25,0.63,0.44}{{#1}}}
    \newcommand{\CharTok}[1]{\textcolor[rgb]{0.25,0.44,0.63}{{#1}}}
    \newcommand{\StringTok}[1]{\textcolor[rgb]{0.25,0.44,0.63}{{#1}}}
    \newcommand{\CommentTok}[1]{\textcolor[rgb]{0.38,0.63,0.69}{\textit{{#1}}}}
    \newcommand{\OtherTok}[1]{\textcolor[rgb]{0.00,0.44,0.13}{{#1}}}
    \newcommand{\AlertTok}[1]{\textcolor[rgb]{1.00,0.00,0.00}{\textbf{{#1}}}}
    \newcommand{\FunctionTok}[1]{\textcolor[rgb]{0.02,0.16,0.49}{{#1}}}
    \newcommand{\RegionMarkerTok}[1]{{#1}}
    \newcommand{\ErrorTok}[1]{\textcolor[rgb]{1.00,0.00,0.00}{\textbf{{#1}}}}
    \newcommand{\NormalTok}[1]{{#1}}

    % Additional commands for more recent versions of Pandoc
    \newcommand{\ConstantTok}[1]{\textcolor[rgb]{0.53,0.00,0.00}{{#1}}}
    \newcommand{\SpecialCharTok}[1]{\textcolor[rgb]{0.25,0.44,0.63}{{#1}}}
    \newcommand{\VerbatimStringTok}[1]{\textcolor[rgb]{0.25,0.44,0.63}{{#1}}}
    \newcommand{\SpecialStringTok}[1]{\textcolor[rgb]{0.73,0.40,0.53}{{#1}}}
    \newcommand{\ImportTok}[1]{{#1}}
    \newcommand{\DocumentationTok}[1]{\textcolor[rgb]{0.73,0.13,0.13}{\textit{{#1}}}}
    \newcommand{\AnnotationTok}[1]{\textcolor[rgb]{0.38,0.63,0.69}{\textbf{\textit{{#1}}}}}
    \newcommand{\CommentVarTok}[1]{\textcolor[rgb]{0.38,0.63,0.69}{\textbf{\textit{{#1}}}}}
    \newcommand{\VariableTok}[1]{\textcolor[rgb]{0.10,0.09,0.49}{{#1}}}
    \newcommand{\ControlFlowTok}[1]{\textcolor[rgb]{0.00,0.44,0.13}{\textbf{{#1}}}}
    \newcommand{\OperatorTok}[1]{\textcolor[rgb]{0.40,0.40,0.40}{{#1}}}
    \newcommand{\BuiltInTok}[1]{{#1}}
    \newcommand{\ExtensionTok}[1]{{#1}}
    \newcommand{\PreprocessorTok}[1]{\textcolor[rgb]{0.74,0.48,0.00}{{#1}}}
    \newcommand{\AttributeTok}[1]{\textcolor[rgb]{0.49,0.56,0.16}{{#1}}}
    \newcommand{\InformationTok}[1]{\textcolor[rgb]{0.38,0.63,0.69}{\textbf{\textit{{#1}}}}}
    \newcommand{\WarningTok}[1]{\textcolor[rgb]{0.38,0.63,0.69}{\textbf{\textit{{#1}}}}}


    % Define a nice break command that doesn't care if a line doesn't already
    % exist.
    \def\br{\hspace*{\fill} \\* }
    % Math Jax compatibility definitions
    \def\gt{>}
    \def\lt{<}
    \let\Oldtex\TeX
    \let\Oldlatex\LaTeX
    \renewcommand{\TeX}{\textrm{\Oldtex}}
    \renewcommand{\LaTeX}{\textrm{\Oldlatex}}
    % Document parameters
    % Document title
    \title{Numerical Solution of Poisson Equation}
    
    
    
    
    
% Pygments definitions
\makeatletter
\def\PY@reset{\let\PY@it=\relax \let\PY@bf=\relax%
    \let\PY@ul=\relax \let\PY@tc=\relax%
    \let\PY@bc=\relax \let\PY@ff=\relax}
\def\PY@tok#1{\csname PY@tok@#1\endcsname}
\def\PY@toks#1+{\ifx\relax#1\empty\else%
    \PY@tok{#1}\expandafter\PY@toks\fi}
\def\PY@do#1{\PY@bc{\PY@tc{\PY@ul{%
    \PY@it{\PY@bf{\PY@ff{#1}}}}}}}
\def\PY#1#2{\PY@reset\PY@toks#1+\relax+\PY@do{#2}}

\@namedef{PY@tok@w}{\def\PY@tc##1{\textcolor[rgb]{0.73,0.73,0.73}{##1}}}
\@namedef{PY@tok@c}{\let\PY@it=\textit\def\PY@tc##1{\textcolor[rgb]{0.24,0.48,0.48}{##1}}}
\@namedef{PY@tok@cp}{\def\PY@tc##1{\textcolor[rgb]{0.61,0.40,0.00}{##1}}}
\@namedef{PY@tok@k}{\let\PY@bf=\textbf\def\PY@tc##1{\textcolor[rgb]{0.00,0.50,0.00}{##1}}}
\@namedef{PY@tok@kp}{\def\PY@tc##1{\textcolor[rgb]{0.00,0.50,0.00}{##1}}}
\@namedef{PY@tok@kt}{\def\PY@tc##1{\textcolor[rgb]{0.69,0.00,0.25}{##1}}}
\@namedef{PY@tok@o}{\def\PY@tc##1{\textcolor[rgb]{0.40,0.40,0.40}{##1}}}
\@namedef{PY@tok@ow}{\let\PY@bf=\textbf\def\PY@tc##1{\textcolor[rgb]{0.67,0.13,1.00}{##1}}}
\@namedef{PY@tok@nb}{\def\PY@tc##1{\textcolor[rgb]{0.00,0.50,0.00}{##1}}}
\@namedef{PY@tok@nf}{\def\PY@tc##1{\textcolor[rgb]{0.00,0.00,1.00}{##1}}}
\@namedef{PY@tok@nc}{\let\PY@bf=\textbf\def\PY@tc##1{\textcolor[rgb]{0.00,0.00,1.00}{##1}}}
\@namedef{PY@tok@nn}{\let\PY@bf=\textbf\def\PY@tc##1{\textcolor[rgb]{0.00,0.00,1.00}{##1}}}
\@namedef{PY@tok@ne}{\let\PY@bf=\textbf\def\PY@tc##1{\textcolor[rgb]{0.80,0.25,0.22}{##1}}}
\@namedef{PY@tok@nv}{\def\PY@tc##1{\textcolor[rgb]{0.10,0.09,0.49}{##1}}}
\@namedef{PY@tok@no}{\def\PY@tc##1{\textcolor[rgb]{0.53,0.00,0.00}{##1}}}
\@namedef{PY@tok@nl}{\def\PY@tc##1{\textcolor[rgb]{0.46,0.46,0.00}{##1}}}
\@namedef{PY@tok@ni}{\let\PY@bf=\textbf\def\PY@tc##1{\textcolor[rgb]{0.44,0.44,0.44}{##1}}}
\@namedef{PY@tok@na}{\def\PY@tc##1{\textcolor[rgb]{0.41,0.47,0.13}{##1}}}
\@namedef{PY@tok@nt}{\let\PY@bf=\textbf\def\PY@tc##1{\textcolor[rgb]{0.00,0.50,0.00}{##1}}}
\@namedef{PY@tok@nd}{\def\PY@tc##1{\textcolor[rgb]{0.67,0.13,1.00}{##1}}}
\@namedef{PY@tok@s}{\def\PY@tc##1{\textcolor[rgb]{0.73,0.13,0.13}{##1}}}
\@namedef{PY@tok@sd}{\let\PY@it=\textit\def\PY@tc##1{\textcolor[rgb]{0.73,0.13,0.13}{##1}}}
\@namedef{PY@tok@si}{\let\PY@bf=\textbf\def\PY@tc##1{\textcolor[rgb]{0.64,0.35,0.47}{##1}}}
\@namedef{PY@tok@se}{\let\PY@bf=\textbf\def\PY@tc##1{\textcolor[rgb]{0.67,0.36,0.12}{##1}}}
\@namedef{PY@tok@sr}{\def\PY@tc##1{\textcolor[rgb]{0.64,0.35,0.47}{##1}}}
\@namedef{PY@tok@ss}{\def\PY@tc##1{\textcolor[rgb]{0.10,0.09,0.49}{##1}}}
\@namedef{PY@tok@sx}{\def\PY@tc##1{\textcolor[rgb]{0.00,0.50,0.00}{##1}}}
\@namedef{PY@tok@m}{\def\PY@tc##1{\textcolor[rgb]{0.40,0.40,0.40}{##1}}}
\@namedef{PY@tok@gh}{\let\PY@bf=\textbf\def\PY@tc##1{\textcolor[rgb]{0.00,0.00,0.50}{##1}}}
\@namedef{PY@tok@gu}{\let\PY@bf=\textbf\def\PY@tc##1{\textcolor[rgb]{0.50,0.00,0.50}{##1}}}
\@namedef{PY@tok@gd}{\def\PY@tc##1{\textcolor[rgb]{0.63,0.00,0.00}{##1}}}
\@namedef{PY@tok@gi}{\def\PY@tc##1{\textcolor[rgb]{0.00,0.52,0.00}{##1}}}
\@namedef{PY@tok@gr}{\def\PY@tc##1{\textcolor[rgb]{0.89,0.00,0.00}{##1}}}
\@namedef{PY@tok@ge}{\let\PY@it=\textit}
\@namedef{PY@tok@gs}{\let\PY@bf=\textbf}
\@namedef{PY@tok@ges}{\let\PY@bf=\textbf\let\PY@it=\textit}
\@namedef{PY@tok@gp}{\let\PY@bf=\textbf\def\PY@tc##1{\textcolor[rgb]{0.00,0.00,0.50}{##1}}}
\@namedef{PY@tok@go}{\def\PY@tc##1{\textcolor[rgb]{0.44,0.44,0.44}{##1}}}
\@namedef{PY@tok@gt}{\def\PY@tc##1{\textcolor[rgb]{0.00,0.27,0.87}{##1}}}
\@namedef{PY@tok@err}{\def\PY@bc##1{{\setlength{\fboxsep}{\string -\fboxrule}\fcolorbox[rgb]{1.00,0.00,0.00}{1,1,1}{\strut ##1}}}}
\@namedef{PY@tok@kc}{\let\PY@bf=\textbf\def\PY@tc##1{\textcolor[rgb]{0.00,0.50,0.00}{##1}}}
\@namedef{PY@tok@kd}{\let\PY@bf=\textbf\def\PY@tc##1{\textcolor[rgb]{0.00,0.50,0.00}{##1}}}
\@namedef{PY@tok@kn}{\let\PY@bf=\textbf\def\PY@tc##1{\textcolor[rgb]{0.00,0.50,0.00}{##1}}}
\@namedef{PY@tok@kr}{\let\PY@bf=\textbf\def\PY@tc##1{\textcolor[rgb]{0.00,0.50,0.00}{##1}}}
\@namedef{PY@tok@bp}{\def\PY@tc##1{\textcolor[rgb]{0.00,0.50,0.00}{##1}}}
\@namedef{PY@tok@fm}{\def\PY@tc##1{\textcolor[rgb]{0.00,0.00,1.00}{##1}}}
\@namedef{PY@tok@vc}{\def\PY@tc##1{\textcolor[rgb]{0.10,0.09,0.49}{##1}}}
\@namedef{PY@tok@vg}{\def\PY@tc##1{\textcolor[rgb]{0.10,0.09,0.49}{##1}}}
\@namedef{PY@tok@vi}{\def\PY@tc##1{\textcolor[rgb]{0.10,0.09,0.49}{##1}}}
\@namedef{PY@tok@vm}{\def\PY@tc##1{\textcolor[rgb]{0.10,0.09,0.49}{##1}}}
\@namedef{PY@tok@sa}{\def\PY@tc##1{\textcolor[rgb]{0.73,0.13,0.13}{##1}}}
\@namedef{PY@tok@sb}{\def\PY@tc##1{\textcolor[rgb]{0.73,0.13,0.13}{##1}}}
\@namedef{PY@tok@sc}{\def\PY@tc##1{\textcolor[rgb]{0.73,0.13,0.13}{##1}}}
\@namedef{PY@tok@dl}{\def\PY@tc##1{\textcolor[rgb]{0.73,0.13,0.13}{##1}}}
\@namedef{PY@tok@s2}{\def\PY@tc##1{\textcolor[rgb]{0.73,0.13,0.13}{##1}}}
\@namedef{PY@tok@sh}{\def\PY@tc##1{\textcolor[rgb]{0.73,0.13,0.13}{##1}}}
\@namedef{PY@tok@s1}{\def\PY@tc##1{\textcolor[rgb]{0.73,0.13,0.13}{##1}}}
\@namedef{PY@tok@mb}{\def\PY@tc##1{\textcolor[rgb]{0.40,0.40,0.40}{##1}}}
\@namedef{PY@tok@mf}{\def\PY@tc##1{\textcolor[rgb]{0.40,0.40,0.40}{##1}}}
\@namedef{PY@tok@mh}{\def\PY@tc##1{\textcolor[rgb]{0.40,0.40,0.40}{##1}}}
\@namedef{PY@tok@mi}{\def\PY@tc##1{\textcolor[rgb]{0.40,0.40,0.40}{##1}}}
\@namedef{PY@tok@il}{\def\PY@tc##1{\textcolor[rgb]{0.40,0.40,0.40}{##1}}}
\@namedef{PY@tok@mo}{\def\PY@tc##1{\textcolor[rgb]{0.40,0.40,0.40}{##1}}}
\@namedef{PY@tok@ch}{\let\PY@it=\textit\def\PY@tc##1{\textcolor[rgb]{0.24,0.48,0.48}{##1}}}
\@namedef{PY@tok@cm}{\let\PY@it=\textit\def\PY@tc##1{\textcolor[rgb]{0.24,0.48,0.48}{##1}}}
\@namedef{PY@tok@cpf}{\let\PY@it=\textit\def\PY@tc##1{\textcolor[rgb]{0.24,0.48,0.48}{##1}}}
\@namedef{PY@tok@c1}{\let\PY@it=\textit\def\PY@tc##1{\textcolor[rgb]{0.24,0.48,0.48}{##1}}}
\@namedef{PY@tok@cs}{\let\PY@it=\textit\def\PY@tc##1{\textcolor[rgb]{0.24,0.48,0.48}{##1}}}

\def\PYZbs{\char`\\}
\def\PYZus{\char`\_}
\def\PYZob{\char`\{}
\def\PYZcb{\char`\}}
\def\PYZca{\char`\^}
\def\PYZam{\char`\&}
\def\PYZlt{\char`\<}
\def\PYZgt{\char`\>}
\def\PYZsh{\char`\#}
\def\PYZpc{\char`\%}
\def\PYZdl{\char`\$}
\def\PYZhy{\char`\-}
\def\PYZsq{\char`\'}
\def\PYZdq{\char`\"}
\def\PYZti{\char`\~}
% for compatibility with earlier versions
\def\PYZat{@}
\def\PYZlb{[}
\def\PYZrb{]}
\makeatother


    % For linebreaks inside Verbatim environment from package fancyvrb.
    \makeatletter
        \newbox\Wrappedcontinuationbox
        \newbox\Wrappedvisiblespacebox
        \newcommand*\Wrappedvisiblespace {\textcolor{red}{\textvisiblespace}}
        \newcommand*\Wrappedcontinuationsymbol {\textcolor{red}{\llap{\tiny$\m@th\hookrightarrow$}}}
        \newcommand*\Wrappedcontinuationindent {3ex }
        \newcommand*\Wrappedafterbreak {\kern\Wrappedcontinuationindent\copy\Wrappedcontinuationbox}
        % Take advantage of the already applied Pygments mark-up to insert
        % potential linebreaks for TeX processing.
        %        {, <, #, %, $, ' and ": go to next line.
        %        _, }, ^, &, >, - and ~: stay at end of broken line.
        % Use of \textquotesingle for straight quote.
        \newcommand*\Wrappedbreaksatspecials {%
            \def\PYGZus{\discretionary{\char`\_}{\Wrappedafterbreak}{\char`\_}}%
            \def\PYGZob{\discretionary{}{\Wrappedafterbreak\char`\{}{\char`\{}}%
            \def\PYGZcb{\discretionary{\char`\}}{\Wrappedafterbreak}{\char`\}}}%
            \def\PYGZca{\discretionary{\char`\^}{\Wrappedafterbreak}{\char`\^}}%
            \def\PYGZam{\discretionary{\char`\&}{\Wrappedafterbreak}{\char`\&}}%
            \def\PYGZlt{\discretionary{}{\Wrappedafterbreak\char`\<}{\char`\<}}%
            \def\PYGZgt{\discretionary{\char`\>}{\Wrappedafterbreak}{\char`\>}}%
            \def\PYGZsh{\discretionary{}{\Wrappedafterbreak\char`\#}{\char`\#}}%
            \def\PYGZpc{\discretionary{}{\Wrappedafterbreak\char`\%}{\char`\%}}%
            \def\PYGZdl{\discretionary{}{\Wrappedafterbreak\char`\$}{\char`\$}}%
            \def\PYGZhy{\discretionary{\char`\-}{\Wrappedafterbreak}{\char`\-}}%
            \def\PYGZsq{\discretionary{}{\Wrappedafterbreak\textquotesingle}{\textquotesingle}}%
            \def\PYGZdq{\discretionary{}{\Wrappedafterbreak\char`\"}{\char`\"}}%
            \def\PYGZti{\discretionary{\char`\~}{\Wrappedafterbreak}{\char`\~}}%
        }
        % Some characters . , ; ? ! / are not pygmentized.
        % This macro makes them "active" and they will insert potential linebreaks
        \newcommand*\Wrappedbreaksatpunct {%
            \lccode`\~`\.\lowercase{\def~}{\discretionary{\hbox{\char`\.}}{\Wrappedafterbreak}{\hbox{\char`\.}}}%
            \lccode`\~`\,\lowercase{\def~}{\discretionary{\hbox{\char`\,}}{\Wrappedafterbreak}{\hbox{\char`\,}}}%
            \lccode`\~`\;\lowercase{\def~}{\discretionary{\hbox{\char`\;}}{\Wrappedafterbreak}{\hbox{\char`\;}}}%
            \lccode`\~`\:\lowercase{\def~}{\discretionary{\hbox{\char`\:}}{\Wrappedafterbreak}{\hbox{\char`\:}}}%
            \lccode`\~`\?\lowercase{\def~}{\discretionary{\hbox{\char`\?}}{\Wrappedafterbreak}{\hbox{\char`\?}}}%
            \lccode`\~`\!\lowercase{\def~}{\discretionary{\hbox{\char`\!}}{\Wrappedafterbreak}{\hbox{\char`\!}}}%
            \lccode`\~`\/\lowercase{\def~}{\discretionary{\hbox{\char`\/}}{\Wrappedafterbreak}{\hbox{\char`\/}}}%
            \catcode`\.\active
            \catcode`\,\active
            \catcode`\;\active
            \catcode`\:\active
            \catcode`\?\active
            \catcode`\!\active
            \catcode`\/\active
            \lccode`\~`\~
        }
    \makeatother

    \let\OriginalVerbatim=\Verbatim
    \makeatletter
    \renewcommand{\Verbatim}[1][1]{%
        %\parskip\z@skip
        \sbox\Wrappedcontinuationbox {\Wrappedcontinuationsymbol}%
        \sbox\Wrappedvisiblespacebox {\FV@SetupFont\Wrappedvisiblespace}%
        \def\FancyVerbFormatLine ##1{\hsize\linewidth
            \vtop{\raggedright\hyphenpenalty\z@\exhyphenpenalty\z@
                \doublehyphendemerits\z@\finalhyphendemerits\z@
                \strut ##1\strut}%
        }%
        % If the linebreak is at a space, the latter will be displayed as visible
        % space at end of first line, and a continuation symbol starts next line.
        % Stretch/shrink are however usually zero for typewriter font.
        \def\FV@Space {%
            \nobreak\hskip\z@ plus\fontdimen3\font minus\fontdimen4\font
            \discretionary{\copy\Wrappedvisiblespacebox}{\Wrappedafterbreak}
            {\kern\fontdimen2\font}%
        }%

        % Allow breaks at special characters using \PYG... macros.
        \Wrappedbreaksatspecials
        % Breaks at punctuation characters . , ; ? ! and / need catcode=\active
        \OriginalVerbatim[#1,codes*=\Wrappedbreaksatpunct]%
    }
    \makeatother

    % Exact colors from NB
    \definecolor{incolor}{HTML}{303F9F}
    \definecolor{outcolor}{HTML}{D84315}
    \definecolor{cellborder}{HTML}{CFCFCF}
    \definecolor{cellbackground}{HTML}{F7F7F7}

    % prompt
    \makeatletter
    \newcommand{\boxspacing}{\kern\kvtcb@left@rule\kern\kvtcb@boxsep}
    \makeatother
    \newcommand{\prompt}[4]{
        {\ttfamily\llap{{\color{#2}[#3]:\hspace{3pt}#4}}\vspace{-\baselineskip}}
    }
    

    
    % Prevent overflowing lines due to hard-to-break entities
    \sloppy
    % Setup hyperref package
    \hypersetup{
      breaklinks=true,  % so long urls are correctly broken across lines
      colorlinks=true,
      urlcolor=urlcolor,
      linkcolor=linkcolor,
      citecolor=citecolor,
      }
    % Slightly bigger margins than the latex defaults
    
    \geometry{verbose,tmargin=1in,bmargin=1in,lmargin=1in,rmargin=1in}
    
    

\begin{document}
    
    \maketitle
    
    

    
    \hypertarget{introduction}{%
\section{Introduction}\label{introduction}}

    \hypertarget{problem-description}{%
\subsection{Problem Description}\label{problem-description}}

The 2D Poisson equation, appears in various fields of applied
mathematics, engineering, and physics. Such fields are electrodynamics,
fluid mechanics, diffusion, and quantum mechanics. This equation is
given by:

\begin{equation}
-\Delta u = f(x,y) \nonumber
\end{equation}

where
\(\Delta u = \nabla^{2} u = \frac{\partial^{2}u}{\partial x^{2}} + \frac{\partial^{2}u}{\partial y^{2}}, f(x,y)\)
is a given fucntion and when \(f(x,y) = 0,\) it is the homogeneous
Laplacian equation. The function \(f(x,y)\) is defined as:

\begin{equation} f(x,y) = 100(x^2 + y^2) \nonumber \end{equation}

Information that is pertintent to solving the problem is the domain
\(\Omega\) that we are solving the equation on, and the Dirichlet
boundary conditions to solve from. This equation is also an elliptical
partial differential equation, based off of all second order linear PDEs
taking the form of
\begin {equation} Au_{xx} + 2Bu_{xy} + Cu_{yy} + Du_{x} + Eu_{y} + Fu + G = 0 \nonumber
\end{equation} And if the coefficients of each term lead to
\(B^2 -AC < 0\), the equation is elliptical. \(A\) is the coefficient of
\(u_{xx}\), \(B\) is the coefficient of \(u_{xy}\), and \(C\) is the
coefficient of \(u_{yy}\). Since the coefficient of the \(u_{xy}\) term
is zero, and the coefficients of \(A\) and \(B\) are 1, it is true that
\(B^2 -AC < 0\), where \$ -1 \textless{} 0\$. Thus, this is an
elliptical PDE.

\textbf{Domain:} The domain \(\Omega\) can be consticted to two
boundaries \((x,y) \in \Omega = (0,1) \times (0,1)\) such that \(x\) and
\(y\) are within the domains of \(0 \leq x \leq 1\) and
\(0 \leq y \leq 1\).

\textbf{Boundary Conditions:} Since a domain \(\Omega\) is defined,
boundary conditions on that square boundary need to be defined so that a
numerical solution for this partial differential equation (PDE) can be
solved. These boundary conditions correspond to the following functions
on the borders of the domain \(\Omega\):

\clearpage

\begin{equation} u(x,0)=\sin(2\pi x), \ \ \ \ \ 0 \leq x \leq 1, \text{ lower border}, \nonumber \end{equation}

\begin{equation} u(x,1)=\sin(2\pi x), \ \ \ \ \ 0 \leq x \leq 1, \text{ upper border}, \nonumber \end{equation}

\begin{equation} u(0,y)=2\sin(2\pi y), \ \ \ \ \ 0 \leq y \leq 1, \text{ left border},\nonumber \end{equation}

\begin{equation} u(1,y)=2\sin(2\pi y), \ \ \ \ \ 0 \leq y \leq 1, \text{ right border}. \nonumber \end{equation}

    \hypertarget{how-can-this-problem-be-solved}{%
\subsection{How can this problem be
solved?}\label{how-can-this-problem-be-solved}}

Since this problem takes the form of a partial differential equation,
it's much more dififcult to solve than an ordinary differential equation
(ODE). However, the finite difference method can be used to solve this
PDE numerically. There are a few steps that need to be taken. First the
domain need to be discreted into a grid of equidistant points, then the
boundary conditions will be defined along that domain, then the finite
difference method will be used to determine a matrix equation
representing a system of equations to find the value of the fucntion
\(u(x,y)\) at whatever point \((x,y)\). After those values for the
discretized points are found, the solution for the homogeneous PDE can
be graphed and error analysis can be determined based upon the Dirichlet
boundary conditions.

    \hypertarget{numerical-analysis-using-python}{%
\section{Numerical Analysis using
Python}\label{numerical-analysis-using-python}}

To begin, some necessary python libraries must be imported to ensure
that the numerical solutions can be graphed and also there are libraries
to compute various facets of numerical linear algebra.

    \begin{tcolorbox}[breakable, size=fbox, boxrule=1pt, pad at break*=1mm,colback=cellbackground, colframe=cellborder]
\prompt{In}{incolor}{82}{\boxspacing}
\begin{Verbatim}[commandchars=\\\{\}]
\PY{c+c1}{\PYZsh{} Necessary Libraries}
\PY{k+kn}{import} \PY{n+nn}{numpy} \PY{k}{as} \PY{n+nn}{np} 
\PY{k+kn}{import} \PY{n+nn}{math}

\PY{c+c1}{\PYZsh{}plotting funcitons}
\PY{o}{\PYZpc{}}\PY{k}{matplotlib} inline
\PY{k+kn}{import} \PY{n+nn}{matplotlib}\PY{n+nn}{.}\PY{n+nn}{pyplot} \PY{k}{as} \PY{n+nn}{plt}
\PY{k+kn}{import} \PY{n+nn}{warnings}
\PY{n}{warnings}\PY{o}{.}\PY{n}{filterwarnings}\PY{p}{(}\PY{l+s+s2}{\PYZdq{}}\PY{l+s+s2}{ignore}\PY{l+s+s2}{\PYZdq{}}\PY{p}{)}
\PY{k+kn}{from} \PY{n+nn}{IPython}\PY{n+nn}{.}\PY{n+nn}{display} \PY{k+kn}{import} \PY{n}{HTML}
\PY{k+kn}{from} \PY{n+nn}{mpl\PYZus{}toolkits}\PY{n+nn}{.}\PY{n+nn}{mplot3d} \PY{k+kn}{import} \PY{n}{axes3d}
\end{Verbatim}
\end{tcolorbox}

    \hypertarget{discretizing-the-domain-omega}{%
\subsection{\texorpdfstring{Discretizing the Domain
\(\Omega\)}{Discretizing the Domain \textbackslash Omega}}\label{discretizing-the-domain-omega}}

    The region \(\Omega = (0,1) \times (0,1)\) needs to be descretized into
a grid of points. This can be done in a number \(N\) steps in the \(x\)
and \(y\) directions. This finite difference can be determined by
getting the difference between the two boundaries of the domain and
diving it by the number of points desired in both directions. Since the
boundaries of x and y are the same, the finite difference should be the
same in both directions.

\begin{equation} h = \frac{1-0}{N} = \frac{1}{N}  \nonumber \end{equation}
Such that x, and y can be plotted as these two functions where \(i\) and
\(j\) are which number of point until \(N\) and \(h\) is the distance
between the equidistant points.

\begin{equation}x[i]=0+ih, \ \ \  i=0,1,...,N, \nonumber \end{equation}

\begin{equation}x[j]=0+jh, \ \ \  j=0,1,...,N,\nonumber \end{equation}

    \begin{tcolorbox}[breakable, size=fbox, boxrule=1pt, pad at break*=1mm,colback=cellbackground, colframe=cellborder]
\prompt{In}{incolor}{83}{\boxspacing}
\begin{Verbatim}[commandchars=\\\{\}]
\PY{n}{N}\PY{o}{=}\PY{l+m+mi}{10}
\PY{n}{h}\PY{o}{=}\PY{l+m+mi}{1}\PY{o}{/}\PY{n}{N}
\PY{n}{x}\PY{o}{=}\PY{n}{np}\PY{o}{.}\PY{n}{arange}\PY{p}{(}\PY{l+m+mi}{0}\PY{p}{,}\PY{l+m+mf}{1.0001}\PY{p}{,}\PY{n}{h}\PY{p}{)}
\PY{n}{y}\PY{o}{=}\PY{n}{np}\PY{o}{.}\PY{n}{arange}\PY{p}{(}\PY{l+m+mi}{0}\PY{p}{,}\PY{l+m+mf}{1.0001}\PY{p}{,}\PY{n}{h}\PY{p}{)}
\PY{n}{X}\PY{p}{,} \PY{n}{Y} \PY{o}{=} \PY{n}{np}\PY{o}{.}\PY{n}{meshgrid}\PY{p}{(}\PY{n}{x}\PY{p}{,} \PY{n}{y}\PY{p}{)}
\PY{n}{fig} \PY{o}{=} \PY{n}{plt}\PY{o}{.}\PY{n}{figure}\PY{p}{(}\PY{p}{)}
\PY{n}{plt}\PY{o}{.}\PY{n}{plot}\PY{p}{(}\PY{n}{x}\PY{p}{[}\PY{l+m+mi}{1}\PY{p}{]}\PY{p}{,}\PY{n}{y}\PY{p}{[}\PY{l+m+mi}{1}\PY{p}{]}\PY{p}{,}\PY{l+s+s1}{\PYZsq{}}\PY{l+s+s1}{ro}\PY{l+s+s1}{\PYZsq{}}\PY{p}{,}\PY{n}{label}\PY{o}{=}\PY{l+s+s1}{\PYZsq{}}\PY{l+s+s1}{Unknown}\PY{l+s+s1}{\PYZsq{}}\PY{p}{)}\PY{p}{;}
\PY{n}{plt}\PY{o}{.}\PY{n}{plot}\PY{p}{(}\PY{n}{X}\PY{p}{,}\PY{n}{Y}\PY{p}{,}\PY{l+s+s1}{\PYZsq{}}\PY{l+s+s1}{ro}\PY{l+s+s1}{\PYZsq{}}\PY{p}{)}\PY{p}{;}
\PY{n}{plt}\PY{o}{.}\PY{n}{plot}\PY{p}{(}\PY{n}{np}\PY{o}{.}\PY{n}{ones}\PY{p}{(}\PY{n}{N}\PY{o}{+}\PY{l+m+mi}{1}\PY{p}{)}\PY{p}{,}\PY{n}{y}\PY{p}{,}\PY{l+s+s1}{\PYZsq{}}\PY{l+s+s1}{go}\PY{l+s+s1}{\PYZsq{}}\PY{p}{,}\PY{n}{label}\PY{o}{=}\PY{l+s+s1}{\PYZsq{}}\PY{l+s+s1}{Boundary Conditions}\PY{l+s+s1}{\PYZsq{}}\PY{p}{)}\PY{p}{;}
\PY{n}{plt}\PY{o}{.}\PY{n}{plot}\PY{p}{(}\PY{n}{np}\PY{o}{.}\PY{n}{zeros}\PY{p}{(}\PY{n}{N}\PY{o}{+}\PY{l+m+mi}{1}\PY{p}{)}\PY{p}{,}\PY{n}{y}\PY{p}{,}\PY{l+s+s1}{\PYZsq{}}\PY{l+s+s1}{go}\PY{l+s+s1}{\PYZsq{}}\PY{p}{)}\PY{p}{;}
\PY{n}{plt}\PY{o}{.}\PY{n}{plot}\PY{p}{(}\PY{n}{x}\PY{p}{,}\PY{n}{np}\PY{o}{.}\PY{n}{zeros}\PY{p}{(}\PY{n}{N}\PY{o}{+}\PY{l+m+mi}{1}\PY{p}{)}\PY{p}{,}\PY{l+s+s1}{\PYZsq{}}\PY{l+s+s1}{go}\PY{l+s+s1}{\PYZsq{}}\PY{p}{)}\PY{p}{;}
\PY{n}{plt}\PY{o}{.}\PY{n}{plot}\PY{p}{(}\PY{n}{x}\PY{p}{,} \PY{n}{np}\PY{o}{.}\PY{n}{ones}\PY{p}{(}\PY{n}{N}\PY{o}{+}\PY{l+m+mi}{1}\PY{p}{)}\PY{p}{,}\PY{l+s+s1}{\PYZsq{}}\PY{l+s+s1}{go}\PY{l+s+s1}{\PYZsq{}}\PY{p}{)}\PY{p}{;}
\PY{n}{plt}\PY{o}{.}\PY{n}{xlim}\PY{p}{(}\PY{p}{(}\PY{o}{\PYZhy{}}\PY{l+m+mf}{0.1}\PY{p}{,}\PY{l+m+mf}{1.1}\PY{p}{)}\PY{p}{)}
\PY{n}{plt}\PY{o}{.}\PY{n}{ylim}\PY{p}{(}\PY{p}{(}\PY{o}{\PYZhy{}}\PY{l+m+mf}{0.1}\PY{p}{,}\PY{l+m+mf}{1.1}\PY{p}{)}\PY{p}{)}
\PY{n}{plt}\PY{o}{.}\PY{n}{xlabel}\PY{p}{(}\PY{l+s+s1}{\PYZsq{}}\PY{l+s+s1}{x}\PY{l+s+s1}{\PYZsq{}}\PY{p}{)}
\PY{n}{plt}\PY{o}{.}\PY{n}{ylabel}\PY{p}{(}\PY{l+s+s1}{\PYZsq{}}\PY{l+s+s1}{y}\PY{l+s+s1}{\PYZsq{}}\PY{p}{)}
\PY{n}{plt}\PY{o}{.}\PY{n}{gca}\PY{p}{(}\PY{p}{)}\PY{o}{.}\PY{n}{set\PYZus{}aspect}\PY{p}{(}\PY{l+s+s1}{\PYZsq{}}\PY{l+s+s1}{equal}\PY{l+s+s1}{\PYZsq{}}\PY{p}{,} \PY{n}{adjustable}\PY{o}{=}\PY{l+s+s1}{\PYZsq{}}\PY{l+s+s1}{box}\PY{l+s+s1}{\PYZsq{}}\PY{p}{)}
\PY{n}{plt}\PY{o}{.}\PY{n}{legend}\PY{p}{(}\PY{n}{loc}\PY{o}{=}\PY{l+s+s1}{\PYZsq{}}\PY{l+s+s1}{center left}\PY{l+s+s1}{\PYZsq{}}\PY{p}{,} \PY{n}{bbox\PYZus{}to\PYZus{}anchor}\PY{o}{=}\PY{p}{(}\PY{l+m+mi}{1}\PY{p}{,} \PY{l+m+mf}{0.5}\PY{p}{)}\PY{p}{)}
\PY{n}{plt}\PY{o}{.}\PY{n}{title}\PY{p}{(}\PY{l+s+sa}{r}\PY{l+s+s1}{\PYZsq{}}\PY{l+s+s1}{Discrete Grid \PYZdl{}}\PY{l+s+s1}{\PYZbs{}}\PY{l+s+s1}{Omega\PYZus{}}\PY{l+s+si}{\PYZob{}h\PYZcb{}}\PY{l+s+s1}{,\PYZdl{} h= }\PY{l+s+si}{\PYZpc{}s}\PY{l+s+s1}{\PYZsq{}}\PY{o}{\PYZpc{}}\PY{p}{(}\PY{n}{h}\PY{p}{)}\PY{p}{,}\PY{n}{fontsize}\PY{o}{=}\PY{l+m+mi}{24}\PY{p}{,}\PY{n}{y}\PY{o}{=}\PY{l+m+mf}{1.08}\PY{p}{)}
\PY{n}{plt}\PY{o}{.}\PY{n}{show}\PY{p}{(}\PY{p}{)}
\end{Verbatim}
\end{tcolorbox}

    \begin{center}
    \adjustimage{max size={0.9\linewidth}{0.9\paperheight}}{project_files/project_7_0.png}
    \end{center}
    { \hspace*{\fill} \\}
    
    \hypertarget{defining-dirichlet-boundary-conditions}{%
\subsection{Defining Dirichlet Boundary
Conditions}\label{defining-dirichlet-boundary-conditions}}

Partial Differential Equations that don't involve time derivatives often
occur in nature and typically don't have initial conditions and the only
thing that is given are boundary conditions, leading to those types of
problems being called boundary-value problems (BVPs).

A boundary type that applies to this specific problem is the Dirichlet
boundary condition, where a function \(u(x,y)\) is defined on the
boundery \(d\Omega\). Dirichlet conditions just give the value of a
function on a boundary. Another form of a boundary condition is the
Neumann boundary condition where the boundary condition is usually the
outward normal derivative (which is proprotional to the inward flux) on
the boundary. The Neumann condition is usually applied for electrostatic
problems and heat flow.

In this case, the Dirichlet conditions are defined as is:

\begin{equation} u_{i0} = u[i,0] = sin(2\pi x[i]), \text{ for } i = 0,\dots,N, \text{lower}, \nonumber \end{equation}

\begin{equation} u_{iN} = u[i,N] = sin(2\pi x[i]), \text{ for } i = 0,\dots,N, \text{upper}, \nonumber \end{equation}

\begin{equation} u_{0j} = u[0,j] = 2sin(2\pi y[j]), \text{ for } j = 0,\dots,N, \text{left}, \nonumber \end{equation}

\begin{equation} u_{Nj} = u[N,j] = 2sin(2\pi x[j]), \text{ for } j = 0,\dots,N, \text{right}, \nonumber \end{equation}

Using pyplot, and the initialization of the points ealier in the mesh
grid, the boundary conditions of \(u(x,y)\) can be plotted on a three
dimensonal grid, which is depicted below.

    \begin{tcolorbox}[breakable, size=fbox, boxrule=1pt, pad at break*=1mm,colback=cellbackground, colframe=cellborder]
\prompt{In}{incolor}{84}{\boxspacing}
\begin{Verbatim}[commandchars=\\\{\}]
\PY{n}{u} \PY{o}{=} \PY{n}{np}\PY{o}{.}\PY{n}{zeros}\PY{p}{(}\PY{p}{(}\PY{n}{N}\PY{o}{+}\PY{l+m+mi}{1}\PY{p}{,} \PY{n}{N}\PY{o}{+}\PY{l+m+mi}{1}\PY{p}{)}\PY{p}{)}

\PY{c+c1}{\PYZsh{} Horizontal boundaries such on the top and bottom of the boundary conditions}
\PY{k}{for} \PY{n}{i} \PY{o+ow}{in} \PY{n+nb}{range} \PY{p}{(}\PY{l+m+mi}{0}\PY{p}{,}\PY{n}{N}\PY{p}{)}\PY{p}{:}
    \PY{n}{u}\PY{p}{[}\PY{n}{i}\PY{p}{,}\PY{l+m+mi}{0}\PY{p}{]}\PY{o}{=}\PY{n}{np}\PY{o}{.}\PY{n}{sin}\PY{p}{(}\PY{l+m+mi}{2}\PY{o}{*}\PY{n}{np}\PY{o}{.}\PY{n}{pi}\PY{o}{*}\PY{n}{x}\PY{p}{[}\PY{n}{i}\PY{p}{]}\PY{p}{)}
    \PY{n}{u}\PY{p}{[}\PY{n}{i}\PY{p}{,}\PY{n}{N}\PY{p}{]}\PY{o}{=}\PY{n}{np}\PY{o}{.}\PY{n}{sin}\PY{p}{(}\PY{l+m+mi}{2}\PY{o}{*}\PY{n}{np}\PY{o}{.}\PY{n}{pi}\PY{o}{*}\PY{n}{x}\PY{p}{[}\PY{n}{i}\PY{p}{]}\PY{p}{)}
\PY{c+c1}{\PYZsh{} Vertical Boundaries such on the left and right of the boundary conditions}
\PY{k}{for} \PY{n}{j} \PY{o+ow}{in} \PY{n+nb}{range} \PY{p}{(}\PY{l+m+mi}{0}\PY{p}{,}\PY{n}{N}\PY{p}{)}\PY{p}{:}
    \PY{n}{u}\PY{p}{[}\PY{l+m+mi}{0}\PY{p}{,}\PY{n}{j}\PY{p}{]}\PY{o}{=}\PY{l+m+mi}{2}\PY{o}{*}\PY{n}{np}\PY{o}{.}\PY{n}{sin}\PY{p}{(}\PY{l+m+mi}{2}\PY{o}{*}\PY{n}{np}\PY{o}{.}\PY{n}{pi}\PY{o}{*}\PY{n}{y}\PY{p}{[}\PY{n}{j}\PY{p}{]}\PY{p}{)}
    \PY{n}{u}\PY{p}{[}\PY{n}{N}\PY{p}{,}\PY{n}{j}\PY{p}{]}\PY{o}{=}\PY{l+m+mi}{2}\PY{o}{*}\PY{n}{np}\PY{o}{.}\PY{n}{sin}\PY{p}{(}\PY{l+m+mi}{2}\PY{o}{*}\PY{n}{np}\PY{o}{.}\PY{n}{pi}\PY{o}{*}\PY{n}{y}\PY{p}{[}\PY{n}{j}\PY{p}{]}\PY{p}{)}
    
\PY{c+c1}{\PYZsh{}Now this information needs to be plotted }
\PY{n}{fig} \PY{o}{=} \PY{n}{plt}\PY{o}{.}\PY{n}{figure}\PY{p}{(}\PY{p}{)}
\PY{n}{ax} \PY{o}{=} \PY{n}{fig}\PY{o}{.}\PY{n}{add\PYZus{}subplot}\PY{p}{(}\PY{l+m+mi}{111}\PY{p}{,} \PY{n}{projection}\PY{o}{=}\PY{l+s+s1}{\PYZsq{}}\PY{l+s+s1}{3d}\PY{l+s+s1}{\PYZsq{}}\PY{p}{)}    
\PY{c+c1}{\PYZsh{}This wireframe is created by the values X and Y that are defined in the mesh}
\PY{n}{ax}\PY{o}{.}\PY{n}{plot\PYZus{}wireframe}\PY{p}{(}\PY{n}{X}\PY{p}{,} \PY{n}{Y}\PY{p}{,} \PY{n}{u}\PY{p}{,} \PY{n}{color}\PY{o}{=}\PY{l+s+s1}{\PYZsq{}}\PY{l+s+s1}{b}\PY{l+s+s1}{\PYZsq{}}\PY{p}{,} \PY{n}{rstride} \PY{o}{=} \PY{n}{N}\PY{p}{,} \PY{n}{cstride}\PY{o}{=}\PY{n}{N}\PY{p}{)}
\PY{n}{ax}\PY{o}{.}\PY{n}{set\PYZus{}xlabel}\PY{p}{(}\PY{l+s+s1}{\PYZsq{}}\PY{l+s+s1}{x}\PY{l+s+s1}{\PYZsq{}}\PY{p}{)}
\PY{n}{ax}\PY{o}{.}\PY{n}{set\PYZus{}ylabel}\PY{p}{(}\PY{l+s+s1}{\PYZsq{}}\PY{l+s+s1}{y}\PY{l+s+s1}{\PYZsq{}}\PY{p}{)}
\PY{n}{ax}\PY{o}{.}\PY{n}{set\PYZus{}zlabel}\PY{p}{(}\PY{l+s+s1}{\PYZsq{}}\PY{l+s+s1}{u}\PY{l+s+s1}{\PYZsq{}}\PY{p}{)}
\PY{n}{plt}\PY{o}{.}\PY{n}{title}\PY{p}{(}\PY{l+s+sa}{r}\PY{l+s+s1}{\PYZsq{}}\PY{l+s+s1}{Boundary Values}\PY{l+s+s1}{\PYZsq{}}\PY{p}{,} \PY{n}{fontsize} \PY{o}{=} \PY{l+m+mi}{24}\PY{p}{,} \PY{n}{y} \PY{o}{=}\PY{l+m+mf}{1.08}\PY{p}{)}
\PY{n}{plt}\PY{o}{.}\PY{n}{show}\PY{p}{(}\PY{p}{)}

\PY{n}{u2} \PY{o}{=} \PY{n}{u}
\end{Verbatim}
\end{tcolorbox}

    \begin{center}
    \adjustimage{max size={0.9\linewidth}{0.9\paperheight}}{project_files/project_9_0.png}
    \end{center}
    { \hspace*{\fill} \\}
    
    \hypertarget{finite-difference-expansion}{%
\subsection{Finite Difference
Expansion}\label{finite-difference-expansion}}

The equation that needs to be solved is:

\begin{equation} -(u_{xx} + u_{yy}) = f(x,y) \nonumber \end{equation}

Using the limit definition of the derivative, an expression for the
second derivative of a function can be expressed in terms of h.

\begin{equation} \delta^{2}_{x} = \frac{1}{h^2}(u(x+h, y) - 2u(x,y) + u(x-h,y)) \nonumber \end{equation}

\begin{equation} \delta^{2}_{y} = \frac{1}{h^2}(u(x, y+h) - 2u(x,y) + u(x,y-h)) \nonumber \end{equation}

Now, these can be expressed in terms of the value of u at discrete point
\((i,j)\) because we have a small sample size and h is the discrete
distance between each point on our finite difference grid:

\begin{equation} \delta^{2}_{x} = \frac{1}{h^2}(u_{i+1j} - 2u_{ij} + u_{i-1j}) \nonumber \end{equation}

\begin{equation} \delta^{2}_{y} = \frac{1}{h^2}(u_{ij+1} - 2u_{ij} + u_{ij-1}) \nonumber \end{equation}

\(u_{ij}\) is the approximation of \(u(x,y)\) at \(x_i\) and \(y_j\).
The Poisson Difference Equation can be expressed like this now based off
of previous equations.

\begin{equation} -(u_{i-1j} + u_{ij-1} -4u_{ij} +u_{ij+1} + u_{i+1j} ) = h^2f_{ij} \nonumber \end{equation}

Equations of this form can be derived for each point \(u_{ij}\) within
the domain \(\Omega\) and a system of equations can be formed in a
system of \((N-1) \times (N-1)\) equations that can be arranged into a
matrix form.

\begin{equation} A\mathbf{u} = \mathbf{r} \nonumber \end{equation}

where \(A\) is an \((N-1)^2\times(N-1)^2\) matrix made up of the
following block tridiagonal structure

\begin{equation}\left(\begin{array}{ccccccc}
T&I&0&0&.&.&.\\
I&T&I&0&0&.&.\\
.&.&.&.&.&.&.\\
.&.&.&0&I&T&I\\
.&.&.&.&0&I&T\\ \end{array}\right) \nonumber \end{equation}

where \(I\) denotes an \(N-1 \times N-1\) identity matrix and \(T\) is
the tridiagonal matrix of the form:

\begin{equation} T=\left(\begin{array}{ccccccc}
-4&1&0&0&.&.&.\\
1&-4&1&0&0&.&.\\
.&.&.&.&.&.&.\\
.&.&.&0&1&-4&1\\
.&.&.&.&0&1&-4\\ \end{array}\right) \nonumber \end{equation}

In human terms, it each value \(u_{ij}\) term draws from it's four
adjacent surrounding points using and uses the finite difference formula
with a defined distance \(h\) between the points. Then that is equated
to the function \(f(x,y)\). The compliment of each point takes the form
of this, drawing from a point above, to the left, to the right, and
below the point \(u_{ij}\), this compliment is then copied through the
diagonal of the matrix \(A\) where \(-4\) is the center.

\begin{equation} \left(\begin{array}{ccccccc}
.&1&.\\
1&-4&1\\
.&1&.\\
\end{array}\right) \nonumber \end{equation}

The intialization of this matrix is shown below through python code.

    \begin{tcolorbox}[breakable, size=fbox, boxrule=1pt, pad at break*=1mm,colback=cellbackground, colframe=cellborder]
\prompt{In}{incolor}{85}{\boxspacing}
\begin{Verbatim}[commandchars=\\\{\}]
\PY{c+c1}{\PYZsh{} A is an (N\PYZhy{}1)\PYZca{}2 x (N\PYZhy{}1)\PYZca{}2 matrix so that matrix A is initialized.}
\PY{n}{N2} \PY{o}{=} \PY{p}{(}\PY{n}{N}\PY{o}{\PYZhy{}}\PY{l+m+mi}{1}\PY{p}{)} \PY{o}{*} \PY{p}{(}\PY{n}{N}\PY{o}{\PYZhy{}}\PY{l+m+mi}{1}\PY{p}{)}
\PY{n}{A} \PY{o}{=} \PY{n}{np}\PY{o}{.}\PY{n}{zeros}\PY{p}{(}\PY{p}{(}\PY{n}{N2}\PY{p}{,}\PY{n}{N2}\PY{p}{)}\PY{p}{)}

\PY{c+c1}{\PYZsh{} Diagonal}
\PY{k}{for} \PY{n}{i} \PY{o+ow}{in} \PY{n+nb}{range} \PY{p}{(}\PY{l+m+mi}{0}\PY{p}{,} \PY{n}{N}\PY{o}{\PYZhy{}}\PY{l+m+mi}{1}\PY{p}{)}\PY{p}{:}
    \PY{k}{for} \PY{n}{j} \PY{o+ow}{in} \PY{n+nb}{range} \PY{p}{(}\PY{l+m+mi}{0}\PY{p}{,} \PY{n}{N}\PY{o}{\PYZhy{}}\PY{l+m+mi}{1}\PY{p}{)}\PY{p}{:}
        \PY{n}{A}\PY{p}{[}\PY{n}{i}\PY{o}{+}\PY{p}{(}\PY{n}{N}\PY{o}{\PYZhy{}}\PY{l+m+mi}{1}\PY{p}{)}\PY{o}{*}\PY{n}{j}\PY{p}{,} \PY{n}{i}\PY{o}{+}\PY{p}{(}\PY{n}{N}\PY{o}{\PYZhy{}}\PY{l+m+mi}{1}\PY{p}{)}\PY{o}{*}\PY{n}{j}\PY{p}{]} \PY{o}{=}\PY{o}{\PYZhy{}}\PY{l+m+mi}{4}

\PY{c+c1}{\PYZsh{} Lower Diagonal}
\PY{k}{for} \PY{n}{i} \PY{o+ow}{in} \PY{n+nb}{range} \PY{p}{(}\PY{l+m+mi}{1}\PY{p}{,} \PY{n}{N}\PY{o}{\PYZhy{}}\PY{l+m+mi}{1}\PY{p}{)}\PY{p}{:}
    \PY{k}{for} \PY{n}{j} \PY{o+ow}{in} \PY{n+nb}{range} \PY{p}{(}\PY{l+m+mi}{0}\PY{p}{,} \PY{n}{N}\PY{o}{\PYZhy{}}\PY{l+m+mi}{1}\PY{p}{)}\PY{p}{:}
        \PY{n}{A}\PY{p}{[}\PY{n}{i}\PY{o}{+}\PY{p}{(}\PY{n}{N}\PY{o}{\PYZhy{}}\PY{l+m+mi}{1}\PY{p}{)}\PY{o}{*}\PY{n}{j}\PY{p}{,} \PY{n}{i}\PY{o}{+}\PY{p}{(}\PY{n}{N}\PY{o}{\PYZhy{}}\PY{l+m+mi}{1}\PY{p}{)}\PY{o}{*}\PY{n}{j} \PY{o}{\PYZhy{}}\PY{l+m+mi}{1}\PY{p}{]}\PY{o}{=}\PY{l+m+mi}{1}
        
\PY{c+c1}{\PYZsh{} Upper Diagonal}
\PY{k}{for} \PY{n}{i} \PY{o+ow}{in} \PY{n+nb}{range} \PY{p}{(}\PY{l+m+mi}{0}\PY{p}{,} \PY{n}{N}\PY{o}{\PYZhy{}}\PY{l+m+mi}{2}\PY{p}{)}\PY{p}{:}
    \PY{k}{for} \PY{n}{j} \PY{o+ow}{in} \PY{n+nb}{range} \PY{p}{(}\PY{l+m+mi}{0}\PY{p}{,} \PY{n}{N}\PY{o}{\PYZhy{}}\PY{l+m+mi}{1}\PY{p}{)}\PY{p}{:}
        \PY{n}{A}\PY{p}{[}\PY{n}{i}\PY{o}{+}\PY{p}{(}\PY{n}{N}\PY{o}{\PYZhy{}}\PY{l+m+mi}{1}\PY{p}{)}\PY{o}{*}\PY{n}{j}\PY{p}{,} \PY{n}{i}\PY{o}{+}\PY{p}{(}\PY{n}{N}\PY{o}{\PYZhy{}}\PY{l+m+mi}{1}\PY{p}{)}\PY{o}{*}\PY{n}{j}\PY{o}{+}\PY{l+m+mi}{1}\PY{p}{]} \PY{o}{=}\PY{l+m+mi}{1}
        
\PY{c+c1}{\PYZsh{} Lower Identity Matrix}
\PY{k}{for} \PY{n}{i} \PY{o+ow}{in} \PY{n+nb}{range} \PY{p}{(}\PY{l+m+mi}{0}\PY{p}{,} \PY{n}{N}\PY{o}{\PYZhy{}}\PY{l+m+mi}{1}\PY{p}{)}\PY{p}{:}
    \PY{k}{for} \PY{n}{j} \PY{o+ow}{in} \PY{n+nb}{range} \PY{p}{(}\PY{l+m+mi}{1}\PY{p}{,} \PY{n}{N}\PY{o}{\PYZhy{}}\PY{l+m+mi}{1}\PY{p}{)}\PY{p}{:}
        \PY{n}{A}\PY{p}{[}\PY{n}{i}\PY{o}{+}\PY{p}{(}\PY{n}{N}\PY{o}{\PYZhy{}}\PY{l+m+mi}{1}\PY{p}{)}\PY{o}{*}\PY{n}{j}\PY{p}{,} \PY{n}{i}\PY{o}{+}\PY{p}{(}\PY{n}{N}\PY{o}{\PYZhy{}}\PY{l+m+mi}{1}\PY{p}{)}\PY{o}{*}\PY{p}{(}\PY{n}{j}\PY{o}{\PYZhy{}}\PY{l+m+mi}{1}\PY{p}{)}\PY{p}{]} \PY{o}{=}\PY{l+m+mi}{1}
        
\PY{c+c1}{\PYZsh{} Upper Identity Matrix }
\PY{k}{for} \PY{n}{i} \PY{o+ow}{in} \PY{n+nb}{range} \PY{p}{(}\PY{l+m+mi}{0}\PY{p}{,} \PY{n}{N}\PY{o}{\PYZhy{}}\PY{l+m+mi}{1}\PY{p}{)}\PY{p}{:}
    \PY{k}{for} \PY{n}{j} \PY{o+ow}{in} \PY{n+nb}{range} \PY{p}{(}\PY{l+m+mi}{0}\PY{p}{,} \PY{n}{N}\PY{o}{\PYZhy{}}\PY{l+m+mi}{2}\PY{p}{)}\PY{p}{:}
        \PY{n}{A}\PY{p}{[}\PY{n}{i}\PY{o}{+}\PY{p}{(}\PY{n}{N}\PY{o}{\PYZhy{}}\PY{l+m+mi}{1}\PY{p}{)}\PY{o}{*}\PY{n}{j}\PY{p}{,} \PY{n}{i}\PY{o}{+}\PY{p}{(}\PY{n}{N}\PY{o}{\PYZhy{}}\PY{l+m+mi}{1}\PY{p}{)}\PY{o}{*}\PY{p}{(}\PY{n}{j}\PY{o}{+}\PY{l+m+mi}{1}\PY{p}{)}\PY{p}{]} \PY{o}{=}\PY{l+m+mi}{1}
        
\PY{n}{Ainv}\PY{o}{=}\PY{n}{np}\PY{o}{.}\PY{n}{linalg}\PY{o}{.}\PY{n}{inv}\PY{p}{(}\PY{n}{A}\PY{p}{)}   
\PY{n}{fig} \PY{o}{=} \PY{n}{plt}\PY{o}{.}\PY{n}{figure}\PY{p}{(}\PY{n}{figsize}\PY{o}{=}\PY{p}{(}\PY{l+m+mi}{12}\PY{p}{,}\PY{l+m+mi}{4}\PY{p}{)}\PY{p}{)}
\PY{n}{plt}\PY{o}{.}\PY{n}{subplot}\PY{p}{(}\PY{l+m+mi}{121}\PY{p}{)}
\PY{n}{plt}\PY{o}{.}\PY{n}{imshow}\PY{p}{(}\PY{n}{A}\PY{p}{,}\PY{n}{interpolation}\PY{o}{=}\PY{l+s+s1}{\PYZsq{}}\PY{l+s+s1}{none}\PY{l+s+s1}{\PYZsq{}}\PY{p}{)}
\PY{n}{clb}\PY{o}{=}\PY{n}{plt}\PY{o}{.}\PY{n}{colorbar}\PY{p}{(}\PY{p}{)}
\PY{n}{clb}\PY{o}{.}\PY{n}{set\PYZus{}label}\PY{p}{(}\PY{l+s+s1}{\PYZsq{}}\PY{l+s+s1}{Matrix elements values}\PY{l+s+s1}{\PYZsq{}}\PY{p}{)}
\PY{n}{plt}\PY{o}{.}\PY{n}{title}\PY{p}{(}\PY{l+s+s1}{\PYZsq{}}\PY{l+s+s1}{Matrix A }\PY{l+s+s1}{\PYZsq{}}\PY{p}{,}\PY{n}{fontsize}\PY{o}{=}\PY{l+m+mi}{24}\PY{p}{)}
\PY{n}{plt}\PY{o}{.}\PY{n}{subplot}\PY{p}{(}\PY{l+m+mi}{122}\PY{p}{)}
\PY{n}{plt}\PY{o}{.}\PY{n}{imshow}\PY{p}{(}\PY{n}{Ainv}\PY{p}{,}\PY{n}{interpolation}\PY{o}{=}\PY{l+s+s1}{\PYZsq{}}\PY{l+s+s1}{none}\PY{l+s+s1}{\PYZsq{}}\PY{p}{)}
\PY{n}{clb}\PY{o}{=}\PY{n}{plt}\PY{o}{.}\PY{n}{colorbar}\PY{p}{(}\PY{p}{)}
\PY{n}{clb}\PY{o}{.}\PY{n}{set\PYZus{}label}\PY{p}{(}\PY{l+s+s1}{\PYZsq{}}\PY{l+s+s1}{Matrix elements values}\PY{l+s+s1}{\PYZsq{}}\PY{p}{)}
\PY{n}{plt}\PY{o}{.}\PY{n}{title}\PY{p}{(}\PY{l+s+sa}{r}\PY{l+s+s1}{\PYZsq{}}\PY{l+s+s1}{Matrix \PYZdl{}A\PYZca{}}\PY{l+s+s1}{\PYZob{}}\PY{l+s+s1}{\PYZhy{}1\PYZcb{}\PYZdl{} }\PY{l+s+s1}{\PYZsq{}}\PY{p}{,}\PY{n}{fontsize}\PY{o}{=}\PY{l+m+mi}{24}\PY{p}{)}

\PY{n}{fig}\PY{o}{.}\PY{n}{tight\PYZus{}layout}\PY{p}{(}\PY{p}{)}
\PY{n}{plt}\PY{o}{.}\PY{n}{show}\PY{p}{(}\PY{p}{)}
\end{Verbatim}
\end{tcolorbox}

    \begin{center}
    \adjustimage{max size={0.9\linewidth}{0.9\paperheight}}{project_files/project_11_0.png}
    \end{center}
    { \hspace*{\fill} \\}
    
    \hypertarget{defining-vectors-textbfu-and-textbfr}{%
\subsection{\texorpdfstring{Defining Vectors \(\textbf{u}\) and
\(\textbf{r}\)}{Defining Vectors \textbackslash textbf\{u\} and \textbackslash textbf\{r\}}}\label{defining-vectors-textbfu-and-textbfr}}

Now, the vector \(\textbf{u}\) is a column vector that is made up of
\(N-1\) subvectors \(\textbf{u}_j\) of N-1 of the form

\begin{equation}\mathbf{u}_j=\left(\begin{array}{c}
u_{1j}\\
u_{2j}\\
.\\
.\\
u_{N-2j}\\
u_{N-1j}\\
\end{array}\right) \nonumber \end{equation}

These subvectors are added together so that there is a column vector of
only the points \(u_{ij}\) for all points inside the domain \(\Omega\).

Also, the solution vector \(\textbf{r}\) of length
\((N-1) \times (N-1)\) is made up of \((N-1)\) subvectors of form
\(\textbf{r}_{j} = -h^2\textbf{f}_{j} -\textbf{bx}_{j} - \textbf{by}_{j}\).
The \(\textbf{b}\) vectors are the boundary conditions on the left and
right. Vector \(\textbf{bx}_{j}\) is the vector of left and right
boundary conditions on a vertical axis:

\begin{equation}\mathbf{bx}_j =\left(\begin{array}{c}
u_{0j}\\
0\\
.\\
.\\
0\\
u_{Nj} \end{array}\right), \nonumber \end{equation}

Also, \(\textbf{by}_1\) can be the lower boundary for \(j = 1\),

\begin{equation}\mathbf{by}_1 =\left(\begin{array}{c}
u_{10}\\
u_{20}\\
.\\
.\\
u_{(N-2)0}\\
u_{(N-1)0} \end{array}\right), \nonumber \end{equation}

The upper boundary can be where \(j = N-1\)

\begin{equation}\mathbf{by}_{N-1} =\left(\begin{array}{c}
u_{1N}\\
u_{2N}\\
.\\
.\\
u_{(N-2)N}\\
u_{(N-1)N} \end{array}\right), \nonumber \end{equation}

Now the boundary conditions are obviously zero when
\(i = 2, \dots, N-2\) and \(j= 2, \dots, N-2\)

\begin{equation} \mathbf{bx}_i, \mathbf{by}_j = 0  \nonumber \end{equation}

Also for the function \(\textbf{f}_{j}\), for the Poisson equation,

\begin{equation}\mathbf{f}_j =100\left(\begin{array}{c}
x_1^2+y_j^2\\
x_2^2+y_j^2\\
.\\
.\\
x_{N-2}^2+y_j^2\\
x_{N-1}^2+y_j^2 \end{array}\right) \nonumber \end{equation} for
\(j=1,...,N-1\).

For the Laplacian equation,

\begin{equation}\mathbf{f}_j = \mathbf{0} \nonumber \end{equation} for
\(j=1, \dots ,N-1\).

The initialization of the vectors \(\textbf{f}_{ij}\) and \(\textbf{r}\)
is in this python code below.

    \begin{tcolorbox}[breakable, size=fbox, boxrule=1pt, pad at break*=1mm,colback=cellbackground, colframe=cellborder]
\prompt{In}{incolor}{86}{\boxspacing}
\begin{Verbatim}[commandchars=\\\{\}]
\PY{n}{f1}\PY{o}{=}\PY{n}{np}\PY{o}{.}\PY{n}{zeros}\PY{p}{(}\PY{n}{N2}\PY{p}{)}
\PY{n}{f2} \PY{o}{=} \PY{n}{np}\PY{o}{.}\PY{n}{zeros}\PY{p}{(}\PY{n}{N2}\PY{p}{)}

\PY{c+c1}{\PYZsh{} vector f1 (Laplacian    }

\PY{c+c1}{\PYZsh{} vector f2 (Poisson 100(x\PYZca{}2+y\PYZca{}2))        }
\PY{k}{for} \PY{n}{i} \PY{o+ow}{in} \PY{n+nb}{range} \PY{p}{(}\PY{l+m+mi}{0}\PY{p}{,} \PY{n}{N}\PY{o}{\PYZhy{}}\PY{l+m+mi}{1}\PY{p}{)}\PY{p}{:}
    \PY{k}{for} \PY{n}{j} \PY{o+ow}{in} \PY{n+nb}{range} \PY{p}{(}\PY{l+m+mi}{0}\PY{p}{,} \PY{n}{N}\PY{o}{\PYZhy{}}\PY{l+m+mi}{1}\PY{p}{)}\PY{p}{:}
        \PY{n}{f2}\PY{p}{[}\PY{n}{i}\PY{o}{+}\PY{p}{(}\PY{n}{N}\PY{o}{\PYZhy{}}\PY{l+m+mi}{1}\PY{p}{)}\PY{o}{*}\PY{n}{j}\PY{p}{]} \PY{o}{=} \PY{l+m+mi}{100}\PY{o}{*}\PY{n}{h}\PY{o}{*}\PY{n}{h}\PY{o}{*}\PY{p}{(}\PY{n}{x}\PY{p}{[}\PY{n}{i}\PY{o}{+}\PY{l+m+mi}{1}\PY{p}{]}\PY{o}{*}\PY{n}{x}\PY{p}{[}\PY{n}{i}\PY{o}{+}\PY{l+m+mi}{1}\PY{p}{]}\PY{o}{+}\PY{n}{y}\PY{p}{[}\PY{n}{j}\PY{o}{+}\PY{l+m+mi}{1}\PY{p}{]}\PY{o}{*}\PY{n}{y}\PY{p}{[}\PY{n}{j}\PY{o}{+}\PY{l+m+mi}{1}\PY{p}{]}\PY{p}{)}  
          
\PY{c+c1}{\PYZsh{} Boundary        }
\PY{n}{b\PYZus{}bottom\PYZus{}top}\PY{o}{=}\PY{n}{np}\PY{o}{.}\PY{n}{zeros}\PY{p}{(}\PY{n}{N2}\PY{p}{)}
\PY{k}{for} \PY{n}{i} \PY{o+ow}{in} \PY{n+nb}{range} \PY{p}{(}\PY{l+m+mi}{0}\PY{p}{,}\PY{n}{N}\PY{o}{\PYZhy{}}\PY{l+m+mi}{1}\PY{p}{)}\PY{p}{:}
    \PY{n}{b\PYZus{}bottom\PYZus{}top}\PY{p}{[}\PY{n}{i}\PY{p}{]}\PY{o}{=}\PY{n}{np}\PY{o}{.}\PY{n}{sin}\PY{p}{(}\PY{l+m+mi}{2}\PY{o}{*}\PY{n}{np}\PY{o}{.}\PY{n}{pi}\PY{o}{*}\PY{n}{x}\PY{p}{[}\PY{n}{i}\PY{o}{+}\PY{l+m+mi}{1}\PY{p}{]}\PY{p}{)} \PY{c+c1}{\PYZsh{}Bottom Boundary}
    \PY{n}{b\PYZus{}bottom\PYZus{}top}\PY{p}{[}\PY{n}{i}\PY{o}{+}\PY{p}{(}\PY{n}{N}\PY{o}{\PYZhy{}}\PY{l+m+mi}{1}\PY{p}{)}\PY{o}{*}\PY{p}{(}\PY{n}{N}\PY{o}{\PYZhy{}}\PY{l+m+mi}{2}\PY{p}{)}\PY{p}{]}\PY{o}{=}\PY{n}{np}\PY{o}{.}\PY{n}{sin}\PY{p}{(}\PY{l+m+mi}{2}\PY{o}{*}\PY{n}{np}\PY{o}{.}\PY{n}{pi}\PY{o}{*}\PY{n}{x}\PY{p}{[}\PY{n}{i}\PY{o}{+}\PY{l+m+mi}{1}\PY{p}{]}\PY{p}{)}\PY{c+c1}{\PYZsh{} Top Boundary}
      
\PY{n}{b\PYZus{}left\PYZus{}right}\PY{o}{=}\PY{n}{np}\PY{o}{.}\PY{n}{zeros}\PY{p}{(}\PY{n}{N2}\PY{p}{)}
\PY{k}{for} \PY{n}{j} \PY{o+ow}{in} \PY{n+nb}{range} \PY{p}{(}\PY{l+m+mi}{0}\PY{p}{,}\PY{n}{N}\PY{o}{\PYZhy{}}\PY{l+m+mi}{1}\PY{p}{)}\PY{p}{:}
    \PY{n}{b\PYZus{}left\PYZus{}right}\PY{p}{[}\PY{p}{(}\PY{n}{N}\PY{o}{\PYZhy{}}\PY{l+m+mi}{1}\PY{p}{)}\PY{o}{*}\PY{n}{j}\PY{p}{]}\PY{o}{=}\PY{l+m+mi}{2}\PY{o}{*}\PY{n}{np}\PY{o}{.}\PY{n}{sin}\PY{p}{(}\PY{l+m+mi}{2}\PY{o}{*}\PY{n}{np}\PY{o}{.}\PY{n}{pi}\PY{o}{*}\PY{n}{y}\PY{p}{[}\PY{n}{j}\PY{o}{+}\PY{l+m+mi}{1}\PY{p}{]}\PY{p}{)} \PY{c+c1}{\PYZsh{} Left Boundary}
    \PY{n}{b\PYZus{}left\PYZus{}right}\PY{p}{[}\PY{n}{N}\PY{o}{\PYZhy{}}\PY{l+m+mi}{2}\PY{o}{+}\PY{p}{(}\PY{n}{N}\PY{o}{\PYZhy{}}\PY{l+m+mi}{1}\PY{p}{)}\PY{o}{*}\PY{n}{j}\PY{p}{]}\PY{o}{=}\PY{l+m+mi}{2}\PY{o}{*}\PY{n}{np}\PY{o}{.}\PY{n}{sin}\PY{p}{(}\PY{l+m+mi}{2}\PY{o}{*}\PY{n}{np}\PY{o}{.}\PY{n}{pi}\PY{o}{*}\PY{n}{y}\PY{p}{[}\PY{n}{j}\PY{o}{+}\PY{l+m+mi}{1}\PY{p}{]}\PY{p}{)}\PY{c+c1}{\PYZsh{} Right Boundary}
    
\PY{n}{b}\PY{o}{=}\PY{n}{b\PYZus{}left\PYZus{}right}\PY{o}{+}\PY{n}{b\PYZus{}bottom\PYZus{}top}

\PY{c+c1}{\PYZsh{}\PYZsh{} Moving the negative sign here}
\PY{n}{r1} \PY{o}{=} \PY{o}{\PYZhy{}}\PY{p}{(}\PY{n}{f1}\PY{o}{+}\PY{n}{b}\PY{p}{)}
\PY{n}{r2} \PY{o}{=} \PY{o}{\PYZhy{}}\PY{p}{(}\PY{n}{f2}\PY{o}{+}\PY{n}{b}\PY{p}{)}
\end{Verbatim}
\end{tcolorbox}

    \hypertarget{numerical-solution}{%
\section{Numerical Solution}\label{numerical-solution}}

Since matrix \(\textbf{A}\) and vector \(\textbf{r}\) are defined, the
system of equations can be solved with a matrix inversion. The original
equation,

\begin{equation} A\mathbf{u} = \mathbf{r} \nonumber \end{equation} can
be inverted to show:

\begin{equation} \mathbf{u} = A^{-1}\mathbf{r} \nonumber \end{equation}

Leaving the solution for \(\textbf{u}\) to be defined. This matrix
inversion and multiplication is done in the python code below.

Also, the solution to \(\textbf{u}(x,y)\) is plotted for both the
Laplacian and the Poisson equation. The Laplacian where \(f(x,y) = 0\)
and the Poisson equation where \(f(x,y) = 100(x^2 +y^2)\).

    \begin{tcolorbox}[breakable, size=fbox, boxrule=1pt, pad at break*=1mm,colback=cellbackground, colframe=cellborder]
\prompt{In}{incolor}{87}{\boxspacing}
\begin{Verbatim}[commandchars=\\\{\}]
\PY{c+c1}{\PYZsh{}Laplacian}
\PY{n}{C}\PY{o}{=}\PY{n}{np}\PY{o}{.}\PY{n}{dot}\PY{p}{(}\PY{n}{Ainv}\PY{p}{,}\PY{n}{r1}\PY{p}{)}
\PY{n}{u}\PY{p}{[}\PY{l+m+mi}{1}\PY{p}{:}\PY{n}{N}\PY{p}{,}\PY{l+m+mi}{1}\PY{p}{:}\PY{n}{N}\PY{p}{]}\PY{o}{=}\PY{n}{C}\PY{o}{.}\PY{n}{reshape}\PY{p}{(}\PY{p}{(}\PY{n}{N}\PY{o}{\PYZhy{}}\PY{l+m+mi}{1}\PY{p}{,}\PY{n}{N}\PY{o}{\PYZhy{}}\PY{l+m+mi}{1}\PY{p}{)}\PY{p}{)}

\PY{n}{fig} \PY{o}{=} \PY{n}{plt}\PY{o}{.}\PY{n}{figure}\PY{p}{(}\PY{n}{figsize}\PY{o}{=}\PY{p}{(}\PY{l+m+mi}{8}\PY{p}{,}\PY{l+m+mi}{6}\PY{p}{)}\PY{p}{)}
\PY{n}{ax} \PY{o}{=} \PY{n}{fig}\PY{o}{.}\PY{n}{add\PYZus{}subplot}\PY{p}{(}\PY{l+m+mi}{111}\PY{p}{,} \PY{n}{projection}\PY{o}{=}\PY{l+s+s1}{\PYZsq{}}\PY{l+s+s1}{3d}\PY{l+s+s1}{\PYZsq{}}\PY{p}{)}
\PY{c+c1}{\PYZsh{} Plot a basic wireframe.}
\PY{n}{ax}\PY{o}{.}\PY{n}{plot\PYZus{}wireframe}\PY{p}{(}\PY{n}{X}\PY{p}{,} \PY{n}{Y}\PY{p}{,} \PY{n}{u}\PY{p}{,}\PY{n}{color}\PY{o}{=}\PY{l+s+s1}{\PYZsq{}}\PY{l+s+s1}{b}\PY{l+s+s1}{\PYZsq{}}\PY{p}{)}
\PY{n}{ax}\PY{o}{.}\PY{n}{set\PYZus{}xlabel}\PY{p}{(}\PY{l+s+s1}{\PYZsq{}}\PY{l+s+s1}{x}\PY{l+s+s1}{\PYZsq{}}\PY{p}{)}
\PY{n}{ax}\PY{o}{.}\PY{n}{set\PYZus{}ylabel}\PY{p}{(}\PY{l+s+s1}{\PYZsq{}}\PY{l+s+s1}{y}\PY{l+s+s1}{\PYZsq{}}\PY{p}{)}
\PY{n}{ax}\PY{o}{.}\PY{n}{set\PYZus{}zlabel}\PY{p}{(}\PY{l+s+s1}{\PYZsq{}}\PY{l+s+s1}{u}\PY{l+s+s1}{\PYZsq{}}\PY{p}{)}
\PY{n}{plt}\PY{o}{.}\PY{n}{title}\PY{p}{(}\PY{l+s+sa}{r}\PY{l+s+s1}{\PYZsq{}}\PY{l+s+s1}{Numerical Approximation of the Laplacian Equation}\PY{l+s+s1}{\PYZsq{}}\PY{p}{,}\PY{n}{fontsize}\PY{o}{=}\PY{l+m+mi}{24}\PY{p}{,}\PY{n}{y}\PY{o}{=}\PY{l+m+mf}{1.08}\PY{p}{)}
\PY{n}{plt}\PY{o}{.}\PY{n}{show}\PY{p}{(}\PY{p}{)}

\PY{c+c1}{\PYZsh{} Poisson}
\PY{n}{C2}\PY{o}{=}\PY{n}{np}\PY{o}{.}\PY{n}{dot}\PY{p}{(}\PY{n}{Ainv}\PY{p}{,}\PY{n}{r2}\PY{p}{)}
\PY{n}{u2}\PY{p}{[}\PY{l+m+mi}{1}\PY{p}{:}\PY{n}{N}\PY{p}{,}\PY{l+m+mi}{1}\PY{p}{:}\PY{n}{N}\PY{p}{]}\PY{o}{=}\PY{n}{C2}\PY{o}{.}\PY{n}{reshape}\PY{p}{(}\PY{p}{(}\PY{n}{N}\PY{o}{\PYZhy{}}\PY{l+m+mi}{1}\PY{p}{,}\PY{n}{N}\PY{o}{\PYZhy{}}\PY{l+m+mi}{1}\PY{p}{)}\PY{p}{)}

\PY{n}{fig2} \PY{o}{=} \PY{n}{plt}\PY{o}{.}\PY{n}{figure}\PY{p}{(}\PY{n}{figsize}\PY{o}{=}\PY{p}{(}\PY{l+m+mi}{8}\PY{p}{,}\PY{l+m+mi}{6}\PY{p}{)}\PY{p}{)}
\PY{n}{ax} \PY{o}{=} \PY{n}{fig2}\PY{o}{.}\PY{n}{add\PYZus{}subplot}\PY{p}{(}\PY{l+m+mi}{111}\PY{p}{,} \PY{n}{projection}\PY{o}{=}\PY{l+s+s1}{\PYZsq{}}\PY{l+s+s1}{3d}\PY{l+s+s1}{\PYZsq{}}\PY{p}{)}
\PY{c+c1}{\PYZsh{} Plot a basic wireframe.}
\PY{n}{ax}\PY{o}{.}\PY{n}{plot\PYZus{}wireframe}\PY{p}{(}\PY{n}{X}\PY{p}{,} \PY{n}{Y}\PY{p}{,} \PY{n}{u2}\PY{p}{,}\PY{n}{color}\PY{o}{=}\PY{l+s+s1}{\PYZsq{}}\PY{l+s+s1}{b}\PY{l+s+s1}{\PYZsq{}}\PY{p}{)}
\PY{n}{ax}\PY{o}{.}\PY{n}{set\PYZus{}xlabel}\PY{p}{(}\PY{l+s+s1}{\PYZsq{}}\PY{l+s+s1}{x}\PY{l+s+s1}{\PYZsq{}}\PY{p}{)}
\PY{n}{ax}\PY{o}{.}\PY{n}{set\PYZus{}ylabel}\PY{p}{(}\PY{l+s+s1}{\PYZsq{}}\PY{l+s+s1}{y}\PY{l+s+s1}{\PYZsq{}}\PY{p}{)}
\PY{n}{ax}\PY{o}{.}\PY{n}{set\PYZus{}zlabel}\PY{p}{(}\PY{l+s+s1}{\PYZsq{}}\PY{l+s+s1}{u}\PY{l+s+s1}{\PYZsq{}}\PY{p}{)}
\PY{n}{plt}\PY{o}{.}\PY{n}{title}\PY{p}{(}\PY{l+s+sa}{r}\PY{l+s+s1}{\PYZsq{}}\PY{l+s+s1}{Numerical Approximation of the Poisson Equation}\PY{l+s+s1}{\PYZsq{}}\PY{p}{,}\PY{n}{fontsize}\PY{o}{=}\PY{l+m+mi}{24}\PY{p}{,}\PY{n}{y}\PY{o}{=}\PY{l+m+mf}{1.08}\PY{p}{)}
\PY{n}{plt}\PY{o}{.}\PY{n}{show}\PY{p}{(}\PY{p}{)}
\end{Verbatim}
\end{tcolorbox}

    \begin{center}
    \adjustimage{max size={0.9\linewidth}{0.9\paperheight}}{project_files/project_15_0.png}
    \end{center}
    { \hspace*{\fill} \\}
    
    \begin{center}
    \adjustimage{max size={0.9\linewidth}{0.9\paperheight}}{project_files/project_15_1.png}
    \end{center}
    { \hspace*{\fill} \\}
    
    \hypertarget{error-analysis}{%
\subsection{Error Analysis}\label{error-analysis}}

    The errors of these functions can be determined by using the numpy
fucntion norm, which calculates the root of squared sum of the elements
in each row.

    \begin{tcolorbox}[breakable, size=fbox, boxrule=1pt, pad at break*=1mm,colback=cellbackground, colframe=cellborder]
\prompt{In}{incolor}{88}{\boxspacing}
\begin{Verbatim}[commandchars=\\\{\}]
\PY{n}{error} \PY{o}{=} \PY{n}{np}\PY{o}{.}\PY{n}{linalg}\PY{o}{.}\PY{n}{norm}\PY{p}{(}\PY{n}{u}\PY{p}{)}
\PY{n+nb}{print}\PY{p}{(}\PY{l+s+s1}{\PYZsq{}}\PY{l+s+s1}{Error (Laplacian) = }\PY{l+s+s1}{\PYZsq{}}\PY{p}{,} \PY{n}{error}\PY{p}{)}

\PY{n}{error2} \PY{o}{=} \PY{n}{np}\PY{o}{.}\PY{n}{linalg}\PY{o}{.}\PY{n}{norm}\PY{p}{(}\PY{n}{u2}\PY{p}{)}
\PY{n+nb}{print}\PY{p}{(}\PY{l+s+s1}{\PYZsq{}}\PY{l+s+s1}{Error (Poisson) = }\PY{l+s+s1}{\PYZsq{}}\PY{p}{,} \PY{n}{error2}\PY{p}{)}
\end{Verbatim}
\end{tcolorbox}

    \begin{Verbatim}[commandchars=\\\{\}]
Error (Laplacian) =  25.676595234851273
Error (Poisson) =  25.676595234851273
    \end{Verbatim}


    % Add a bibliography block to the postdoc
    
    
    
\end{document}
